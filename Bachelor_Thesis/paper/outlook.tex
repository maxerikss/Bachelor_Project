\section{Outlook}
The field of measurement and feedback based cooling seems to be a promising way to cool quantum systems. In this thesis a QHO has been considered, but it is not the only system that in theory could be cooled in this way. However, it has served as a good proof of concept, and has application that are relevant. As mentioned in the discussion, implementing detectors with finite bandwidth is an important consideration for future research as this would more accurately model a real system. 

Furthermore, this thesis only consider one type of feedback scheme, looking at the feedback Hamiltonian as a linear modification of the position and momentum quadratures. There is promising work to be done either considering modifying the position of the potential akin to \cite{De-Sousa:2025}, or modifying the frequency of the oscillator similar to \cite{Habibi:2016}. The latter would be interesting as it seems to have a more direct application in quantum optomechanics.

This thesis has also mainly looked at feedback with the purpose of cooling the system, with the stability secondary. For further research it would be interesting to look at the stability as the main point of interest. There could also be interesting to look at feedback as a way to prepare different types of states which can later be used for other experiments, for example squeezed states.
