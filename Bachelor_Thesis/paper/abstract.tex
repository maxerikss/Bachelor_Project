\section*{\centering Abstract}
Quantum mechanical technologies often relies on the system being cooled to low temperatures. One way to achieve this is with measurement and feedback based cooling. In this thesis measurement and feedback on a quantum harmonic oscillator coupled to a thermal reservoir is considered. The Wiseman-Milburn equation \cite{Annby-Andersson:2024} was used with a linear feedback control to find the steady-state solutions for the first and second moments of the system. The solutions were performed analytically. The results show that there is a feedback parameter such that the system is cooled below the temperature of the thermal reservoir, more generally the magnitude and phase of the feedback parameter can control the temperature in both directions. It is also shown that the system is stable for such a parameter. There also exist feedback parameters which can cool the system, but is unstable, as well as feedback parameters which break one or more of the approximations and assumptions made about the system. Furthermore, it also looks promising using feedback as a way to stabilize an unstable system, since the stability of the system is heavily dependent on the feedback parameter.