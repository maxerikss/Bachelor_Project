\section{Introduction}
As our society is becoming increasingly dependent on technology, the demand for better and more efficient technologies is growing. One kind of technology which has been really important since its conception during the second world war by Alan Turing is computers \cite{Nielsen:2010}. The field of computer science has been researched a lot during the second half of the 20th century but has hit a fundamental problem. Our current classical computers have reached a limit where the transistors cannot get any smaller without quantum mechanics causing issues \cite{Nielsen:2010}. This has prompted research into quantum technologies such as quantum computers and quantum simulations \cite{Nielsen:2010}. 

To properly understand how quantum mechanical systems work, it is important to understand what happens to them when they are interacted with, during for example a measurement \cite{Jordan:2024}. It is also this interaction with a quantum system, which has prompted many interpretations of quantum mechanics and given rise to what is known as the measurement problem \cite{Jordan:2024}. The measurement problem is a fundamental philosophical problem in quantum mechanics, which arises since a quantum mechanical state evolves deterministically according to the Schrödinger equation, but collapses probabilistically when measured or interacted with \cite{Jordan:2024}.

It is also interesting to see how such a system can be manipulated to create a certain state which can be used for a specific purpose. This could include, but is not limited to, creating a qubit state to be used in a quantum computer or a state which can simulate a certain physical system \cite{Nielsen:2010}. Here it is important to understand the effect feedback has on a measured system, and how to utilize this to create a desired state \cite{Annby-Andersson:2024}.

This thesis will look at a quantum harmonic oscillator which is coupled to an environment. Thus creating an open quantum system whose state is temperature dependent. The system will be measured continuously using weak measurements with a feedback loop to control the system. The goal is to see how the system evolves under these conditions, and how the feedback loop can be changed and manipulated to observe different behaviours.

\subsection{Outline}
This text will start in section 2 by introducing the theoretical framework central in this thesis by first defining what we mean by a quantum harmonic oscillator as well as shortly introducing the density matrix formalism of quantum mechanics. Then, we will move on to discuss open quantum systems and the mathematical framework for the evolution of this type of system, and here we will define a Markovian master equation in Lindblad form. We then move on to discussing measurements on open systems as well as feedback control, and how these concepts can be introduced in the master equation to allow for a mathematical description of the evolution of the system under these effects. In section 3 we will use what has been discussed to derive equations of motion for the QHO. In section 4 there will be a discussion of the relevancy of the results obtained in section 3.


