\section{Introduction}
As our society is becoming increasingly dependent on technology, the demand for better and more efficient technologies is growing. One kind of technology which has been really important since its conception during the second world war by Alan Turing is computers \cite{Nielsen:2010}. The field of computer science has been researched a lot during the second half of the 20th century but has hit a fundamental problem. Our current classical computers have reached a limit where the transistors cannot get any smaller without quantum mechanics causing issues \cite{Nielsen:2010}. This has prompted research into quantum technologies such as quantum computers and quantum simulations \cite{Nielsen:2010}. To properly understand how quantum mechanical systems work, it is important to understand what happens to them when they are interacted with, during for example a measurement \cite{Jordan:2024}. It is also interesting to see how such a system can be manipulated to create a certain state which can be used for a specific purpose. 

\subsection{Outline}
This thesis will look at a quantum harmonic oscillator which is coupled to an environment. Thus creating an open quantum system whose state is temperature dependent. The system will be measured continuously using weak measurements with a feedback loop to control the system. The goal is to see how the system evolves under these conditions, and how the feedback loop can be changed and manipulated to observe different behaviours.

\cite{Annby-Andersson:2024}

