\section*{Populärvetenskaplig sammanfattning}
Kvantmekaniken har sedan början av 1900-talet revolutionerat vår försåelse av fysik, framförallt hur ljus och partiklar fungerar och interagerar. En unik egenskap av kvantmekaniken är att den i grunden är en probibalistisk teori, vilket låter oss beskriva system som befinner sig i flera olika tillstånd samtidigt. Detta fenomenet kallas för superposition och är vad som gör kvantmekanik unikt från klassisk fysik. 

En konsekvens av att kvantmekaniska system kan befinna sig i superposition är att mätningar av systemet ger olika resultat fördelade enligt en sannolikhetsfördelning. Detta öppnar också upp för frågan vad en kvantmekanisk mätning är för något. I grund och botten kan vi säga att en mätning är en interaktion mellan vårt system och ett yttre system, en detektor. Det är denna interaktion som gör att systemet kollapsar från en superposition till det observerade tillståndet. En mätning, eller interaktion, påverkar alltså vårt kvantmekaniska system. I detta arbete undersöker vi hur man kan använda mätningar och återkoppling för att kyla ett kvantmekaniskt system.

Det system som vi studerar är ett av de simplaste kvantmekaniska systemen, en kvantharmonisk oscillator. Det är ett av få systm som kan lösas analytiskt, vilket är en av anledningarna till att vi väljer att studera det. En annan fördel med en en kvantharmonisk oscillator är att det är en bra approximation av många system som befinner sig nära sitt jämviktsläge. I modellen som används betraktar vi även ett värmebad som systemet är kopplat till, vilket är ett sätt att modellera en omgivning som påverkar systemet och ger oss en mer realistisk bild.

Vi undersöker sedan hur svaga mätningar påverkar systemet. Med en svag mätning menar vi att istället för att kollapsa systemet helt till ett tilstånd och förstöra superpositionen så får vi ut en begränsad mängd information om systemet, samtidigt som systemet fortfarande är i superposition. Vi matar sedan in denna information i ett återkopplingssystem, som i sin tur påverkar systemet. Genom att justera styrkan och fasen av vår återkopplingsparameter kan vi kyla systemet genom att minimera dessa fluktuationer. Det intressanta resultatet visar att det är möjligt att kyla systemet till en temperatur som är lägre än den i värmebadet. Det vill säga att systemet blir kallare än sin omgivning. Vi visar även i arbetet att systemet är stabilt under återkopplingen som kan kyla systemet. Detta är viktigt för att kunna vara prakiskt tillämpbart då ett instabilt system hade varit väldigt känsligt för störningar, något som alltid finns i verkligheten. Det öppnar också upp för möjligheten att istället för att se kylning som det primära funktionen för återkopplingen kan vi istället tänka oss en återkoppling som försöker stabilisera ett annars instabilt system.

En anledning att vi vill kunna kyla kvantystem är för att kunna observera och utnyttja kvantmekaniska fenomen. Exempelvis kan höga temperaturer excitera systemet till högre energinivåer, och om energinivåer är ett av de tillstånden som är viktiga för vår applikation så kan det leda till brus och informationsförluster. Ett exemepel skulle kunna vara kvantdatorer där vi utnyttjar superposition för att utföra beräkningar. Detta bygger på användandet av kvantbitar, som är en superposition av två tillstånd. Ofta kan den fysiska realiseringen av en kvantbit vara två energinivåer i ett system, där grundtillståndet motsvarar 0 och det första exciterade tillståndet motsvarar 1. Här är det lätt att se att om systemet exciteras av vår omgivning så förlorar vi den informationen som är lagrad i systemet. Kylning av kvantsystem är därför en viktig del för framtida forskning inom kvantteknologier.