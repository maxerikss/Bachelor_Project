\section{Theoretical Framework}
\subsection{Quantum Harmonic Oscillator}
The quantum harmonic oscillator (QHO) is a quantum mechanical system useful for many applications. This stems from the fact that many systems can be approximated as harmonic close to their equilibrium position, and since the QHO is a simple system, which is possible to solve analytically it is a good starting approximation. The system has the Hamiltonian
\begin{equation}
    \hamiltonian = \frac{\pop^2}{2m} + \frac{1}{2}m\omega^2\xop^2 = \hbar\omega\left(\ad\a + \frac{1}{2}\right),
\end{equation}
where $\pop$ and $\xop$ are the momentum and position operators, $m$ is the mass of the particle, and $\omega$ is the angular frequency of the oscillator. The operators $\ad$ and $\a$ are the creation and annihilation operators, which are defined as
\begin{equation}
    \a = \sqrt{\frac{m\omega}{2\hbar}} \left(\xop + \frac{i}{m\omega} \pop\right) \quad \text{and} \quad \ad = \sqrt{\frac{m\omega}{2\hbar}} \left(\xop - \frac{i}{m\omega} \pop\right).
\end{equation}
It is also reasonable to mention the number operator $\hat{n} = \ad\a$ which has the number states, or Fock states, $\ket{n}$ as its eigenstates.

\subsection{Open Quantum Systems}
With an open quantum system we mean a quantum system which in some ways interact with an environment. This interaction could be described as a thermal coupling between the main system and some temperature bath. This will cause the system to be in a thermal equilibrium with the environment if left alone, and therefore it will be dependent on the temperature. Unless the temperature of the bath is zero, the system will be in a mixed state, described by a density matrix $\dmatrix$. Notably, if the temperature is zero, it is equivalent of considering a closed quantum system, since the coupling is thermal. 

The thermal coupling will lead to dissipation of quantum information from the system to the environment. During this process, the system loses coherence. That is, the quantum mechanical properties of the system are lost and a classical description of the state becomes more appropriate. The coherence of the system is manifested in the off-diagonal elements of the density matrix. If the off-diagonal elements are zero, either by dissipation to the environment or by other means of decoherence, the system will exist in a classical probabilistic state, and any superposition of states will be lost. 

\subsubsection{Master Equation}
The evolution of an open quantum system can be described by a master equation, which is a differential equation and generalization of the Schrödinger equation to involve open quantum systems instead of pure states. By introducing the super operator
\begin{equation}
    \super{\hat{L}_k}\dmatrix = \hat{L}_k\dmatrix\hat{L}_k^{\dagger} - \frac{1}{2} \acomm{\hat{L}_k^\dagger \hat{L}_k}{\dmatrix},
\end{equation} 
where $\hat{L}_k$ are called Lindblad operators, the master equation on Lindblad form can be written as 
\begin{equation}
    \dt\dmatrix = -\frac{i}{\hbar}\comm{\hamiltonian}{\dmatrix} + \sum_k \gamma_k\super{\hat{L}_k}\dmatrix,
\end{equation}
where $\hamiltonian$ is the hamiltonian of the system, and $\gamma_k$ are the decay rates of the system, relating the decoherence to the environment depending on the coupling to the system. If $\gamma_k = 0$ for all $k$ the equation reduces to the von Neumann equation for a closed quantum system. The remaining term thus describes the unitary time evolution of the system and is the analogue of the Schrödinger equation for the density matrix formalism.


\subsection{Continuous Measurements}
\subsection{Feedback Control}
\subsection{Wigner Function}