\section{Discussion and Conclusions}

\subsection{Cooling of a Quantum Harmonic Oscillator}
Technologies relying on quantum phonemona such as quantum computers, generally have a need for a low temperature system \cite{Nielsen:2010}. It is therefore an interesting and active research topic to find ways to cool a quantum system. Currently there exist multiple types of cooling, and the work perfomed in this thesis is in the form of measurement and feedback based cooling, similar to \cite{De-Sousa:2025}.

As briefely mentioned in Sec. \ref{sec:feedback}, the physical realisation of the feedback scheme could be a driving laser acting on the system. The successfullness of the cooling would then also depend on the possibilites to physically control the parameters. Especially the feedback strength and phase. It is especially apparent from the results in Fig. \ref{fig:energyFeedbackRatio} that the phase plays a large role in if the feedback is successful, which can be seen from quadrant 2 and 4 having negative energy, which is not physical. The explanation for this is that the system most likely breaks down and for a steady state to exist given these parameters the temperature would have to be negative. Thus, the only physically relevant solutions is in quadrant 1 and 3. We can then also conclude that in general a functionate feedback scheme has a control hamiltonian which is either proportional to $\xop - \pop$ or $-\xop + \pop$. It is possible to have a feedback hamiltonian proportional to $-\xop - \pop$ but this will not be as effective at cooling the system. In panel \textbf{b} in Fig. \ref{fig:energyFeedbackRatio} this area can be seen for $\re{\tilde{f}} < 0$. So even if it is possible to cool in this area, the same amplitude but smaller phase would yield a better result.

The feedback employed in this thesis takes the form of an external force acting on the system, for example a laser interacting with the field inside a cavity, which linearly changes the system hamiltonian. A different way to cool the system would be to modify the frequency of the oscillator \cite{Habibi:2016}, which has applications in quantum optomechanics. Quantum optomechanics has the ability to use light to prepare macroscopic objects in quantum states \cite{Chen:2013}, allowing for quantum mechanical control over macroscopic systems, which have application in for example LIGO. Changing the frequence could prove more difficult to model since the feedback would be non-linear. It's therefore interesting to ask if the results in this thesis, and the feedback scheme used, could be realized in optomechanical systems. 

We only consider the case of infinite detector bandwidth and instantaneous feedback in this thesis. In reality, this idealized picture would not hold, and the effectivness of the feedback scheme might decrease \cite{Annby-Andersson:2024}, affecting the efficiency of the cooling. The addition of finite bandwidth is a new addition in the field of quantum feedback control, with \cite{De-Sousa:2025} being one of the first to consider this. 

Increasing the quality factor of the system has a positive impact on the effectivness of the cooling, allowing smaller feedback parameters to have greater control of the system. A consequence however, is that the system will also be more sensitive to the feedback and measurement. Thus, if the parameters used are prone to noise, the system will vary more than with a lower quality factor.

To get an accurate conclusion about the usefulness of the feedback scheme, it is important to conisder the stability of the system in the physical regions. Thus, it is not certain that just because there exist a steady state solution, we have a stable solution at that point.



\subsection{Stability with Feedback}
The stability of a quantum system is of importance when considering applications based on said system. That is, a stable system can be used and its properties exploited, while an unstable system might behave unpredicatbly, and cause issues. Specifically, we call a system stable, if in the long time limit it decays to a specific state, and does not diverge from this if affected by a small disturbance, barring any external factors which may affect the stability. In the case of the system modeled in this thesis it means that the expectation value of the variance in the sytem decays to a finite value.

We can see from the results in Fig. \ref{fig:eigenvalueFirstMomenta} that the stability of the system is highly dependent on the feedback parameter. It is also important to note that when looking at Fig. \ref{fig:energyFeedbackRatio} only quadrant 1 is in the region of stability of the system. Thus, even if quadrant 3 cools the system, the state will not be stable and it would be difficult to isolate the system from external fluctuations and keep it in this state. This essentially solidifies the fact that the only relevant feedback parameters are those which are in quadrant 1. 


\subsection{Applications}
One interesting application of QHOs is in the field of quantum computation. A particular application is the use of QHOs in conjunction with qubits to create higher dimensional systems, qudits, to be used for calculations \cite{Liu:2021}. A problem noted in \cite{Liu:2021} is that when coupled to a bath, which all physical system are, the qudit will experience noise due to the fluctuations of the bath, which would require quantum error correction to combat \cite{Liu:2021}. However, with too much noise these protocols could fail, and thus cooling the system may help by reducing the noise.

Another interesting use for QHOs is as thermal baths themselves inside quantum circuits as described by \cite{Pekola:2024}. In order to realize this thermalization an ensamble of QHOs would be needed with variable energies \cite{Pekola:2024}. The results in this thesis show that a possible protocol to control the temperature of a QHO which therefore would allow QHOs to be used in this way. Quantum circuits themselves are interesting since they are a way to physically realize a quantum computer \cite{Nielsen:2010}. 



Quantum sensing\\
